\section{Research Experience}

% \vspace{2pt}

%\begin{itemize}

{\cventry{Feb 2022 -- Aug 2022 }{\href{https://github.com/Siddhant-Ray/Network-Traffic-Transformer}{Advancing Packet-Level Traffic Predictions with Transformers}}{\href{https://nsg.ee.ethz.ch/fileadmin/user_upload/thesis_proposal_packet_transformer.pdf}{Master Thesis at the Networked Systems Group}}{ETH Zurich}{}{%\vspace{3pt} 
\begin{itemize}
\item Transformer based neural network learn network dynamics and make smarter packet-level decisions.
\item Used both packet and network state features to capture the underlying patterns in network traffic with a task-agnostic pre-training phase.
\item Followed by task-specific fine-tuning phases to leverage this learnt behaviour, make quicker updates and make better forwarding decisions.
\end{itemize}
}}

{\cventry{Mar 2021 -- Jun 2021}{\href{https://github.com/Siddhant-Ray/FRR-P4-Super-Node-Prototype}{Towards a New Framework for Integration of Network Planes}}{\href{https://nsg.ee.ethz.ch/home/}{Research Project at the Networked Systems Group}}{ETH Zurich}{}{%\vspace{3pt} 
\begin{itemize}
\item Prototype for a new programmble forwarding node (a Super-Node) for dynamic control over forwarding in Layer-3 networks.
\item Super-Node retains a traditional L3 router's control plane (CP) and combines a P4 programmable data plane (DP) for forwarding packets.
\item Accelerated forwarding and network convergence by leveraging the newly combined CP and DP.
\end{itemize}
}}

{\cventry{Feb 2021 -- May 2021}{\href{https://github.com/Siddhant-Ray/Attentive-neural-networks-for-news-classification}{Attentive Neural Networks for News Classification}}{\href{https://www.mins.ee.ethz.ch/index.html}{Research Project at the Chair for Mathematical Information Science}}{ETH Zurich}{}{%\vspace{3pt} 
\begin{itemize}
\item Transformer based neural network to classify a multi-class hierarchical, context-overlapping news dataset. 
\item Created a new statistical algorithm to detect context overlap in dataset classes, used it to reduce class label redundancy.
\item We demonstrated improvements of the classification model based on the algorithm's reduction.
\end{itemize}
}}

{\cventry{Dec 2019 -- May 2020}{\href{https://github.com/Siddhant-Ray/Machine-Learning-Based-Cell-Association}{Machine Learning based Cell Association for 5G Communication Networks}}{\href{https://vit.ac.in/school/facilities/sense}{Bachelor Thesis at the Networking Lab}}{VIT Vellore}{}{%\vspace{3pt} 
\begin{itemize}
\item New cell association scheme to meet the ultra low latency, higher load and traffic needs of the 5G networks.
\item Hidden Markov Model based learning algorithm followed by a Viterbi based decoding scheme, on the network’s telemetry data, to learn network parameters and select the optimal eNodeB for cell association.
 \end{itemize}
}}


%\vspace{4pt}


%\end{itemize}