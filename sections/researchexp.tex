\section{Research Experience}

% \vspace{2pt}

\begin{itemize}

\item{\cventry{Feb 2022 -- Aug 2022 }{\href{https://nsg.ee.ethz.ch/fileadmin/user_upload/thesis_proposal_packet_transformer.pdf}{Master thesis at the Networked Systems Group }}{\href{https://github.com/Siddhant-Ray/Network-Traffic-Transformer}{Advancing Packet-Level Traffic Predictions with Transformers}}{ ETH Zurich}{}{\vspace{3pt} 
\begin{itemize}
\item Developed a Transformer based neural network which can efficiently learning network dynamics and make smarter packet-level forwarding decisions.
\item Using features from both the packet and network state, we capture the underlying patterns in the traffic in a task-agnostic pre-training phase.
\item Followed by task-specific fine-tuning phases to leverage this learnt behaviour, make quicker updates and make better decisions.
\end{itemize}
}}

\item{\cventry{Mar 2021 -- Jun 2021}{\href{https://nsg.ee.ethz.ch/home/}{Research project at the Networked Systems Group }}{\href{https://github.com/Siddhant-Ray/FRR-P4-Super-Node-Prototype}{Towards a New Framework for Integration of Network Planes }}{ ETH Zurich}{}{\vspace{3pt} 
\begin{itemize}
\item Created a prototype for a new forwarding node in programmable networks i.e. a Super-Node to have better and more dynamic  control over forwarding decision in Layer-3 (L3) routers.
\item Our Super-Node consists of a traditional L3 router's control plane (CP) combined with a P4 programmable data plane (DP) for forwarding packets.
\item We attempt accelerated forwarding and network convergence by leveraging the newly combined CP and DP.
\end{itemize}
}}

\item{\cventry{Feb 2021 -- May 2021}{\href{https://www.mins.ee.ethz.ch/index.html}{Research project at the Chair for Mathematical Information Science  }}{\href{https://github.com/Siddhant-Ray/Attentive-neural-networks-for-news-classification}{Attentive Neural Networks for News Classification}}{ ETH Zurich}{}{\vspace{3pt} 
\begin{itemize}
\item Developed a Transformer based neural network to classify a multi-class hierarchical, context-overlapping news dataset. 
\item Created a new statistical algorithm to detect context overlap in dataset classes, used it to reduce class label redundancy and
\item We demonstrated improvements of the classification model based on the algorithm's reduction.
\end{itemize}
}}

\item{\cventry{Dec 2019 -- May 2020}{\href{https://vit.ac.in/school/facilities/sense}{Bachelor thesis at the Networking Lab}}{\href{https://github.com/Siddhant-Ray/Machine-Learning-Based-Cell-Association}{Machine Learning based Cell Association for 5G Communication Networks}}{ VIT Vellore}{}{\vspace{3pt} 
\begin{itemize}
\item Proposed a new cell association scheme to meet the ultra low latency, higher load and traffic needs of the 5G networks.
\item Proposed a Hidden Markov Model based learning algorithm followed by a Viterbi based decoding scheme, on the network’s telemetry data, to learn network parameters and select the optimal eNodeB for cell association.
 \end{itemize}
}}



\vspace{4pt}


% \item{\cventry{June 2018 -- Aug 2018}{\href{https://cutt.ly/MnJxdE3}{ResearchGate}}{A Comparative Analysis and Testing of Supervised ML Algorithms}{BlueStacks}{}{\vspace{3pt}
% \begin{itemize}
% \item  Studied and compared various supervised learning models and attempt to classify the best performing one based on varying comparison metrics
% \item This project was carried out on data captured from front-end customer preference analytics from the BlueStacks engine and use to develop a model to provide personalised customer experience.
% \end{itemize}
% }}

\end{itemize}