%% start of file `template.tex'.
% This work may be distributed and/or modified under the
% conditions of the LaTeX Project Public License version 1.3c,
% available at http://www.latex-project.org/lppl/.


\documentclass[11pt,a4paper,sans]{moderncv}        % possible options include font size ('10pt', '11pt' and '12pt'), paper size ('a4paper', 'letterpaper', 'a5paper', 'legalpaper', 'executivepaper' and 'landscape') and font family ('sans' and 'roman')

% modern themes
\moderncvstyle{banking}                            % style options are 'casual' (default), 'classic', 'oldstyle' and 'banking'
\moderncvcolor{black}                                % color options 'blue' (default), 'orange', 'green', 'red', 'purple', 'grey' and 'black'
%\renewcommand{\familydefault}{\sfdefault}         % to set the default font; use '\sfdefault' for the default sans serif font, '\rmdefault' for the default roman one, or any tex font name
\nopagenumbers{}                                  % uncomment to suppress automatic page numbering for CVs longer than one page

% character encoding
\usepackage[utf8]{inputenc}                       % if you are not using xelatex ou lualatex, replace by the encoding you are using
%\usepackage{CJKutf8}                              % if you need to use CJK to typeset your resume in Chinese, Japanese or Korean
\usepackage{multicol}
% adjust the page margins
%\usepackage[scale=0.8]{geometry}
 \usepackage[left=1.2cm, right=1.2cm, top=0.5cm]{geometry}
%\setlength{\hintscolumnwidth}{3cm}                % if you want to change the width of the column with the dates
%\setlength{\makecvheadnamewidth}{10cm}           % for the 'classic' style, if you want to force the width allocated to your name and avoid line breaks. be careful though, the length is normally calculated to avoid any overlap with your personal info; use this at your own typographical risks...
\usepackage{anyfontsize}
\usepackage{import}
%\usepackage{hyperref}
\usepackage{bibentry}
\usepackage{natbib}

\newenvironment{tightemize}{
\vspace{-\topsep}\begin{itemize}\itemsep1pt \parskip0pt \parsep0pt}
{\end{itemize}\vspace{-\topsep}}

% personal data
\name{Siddhant}{Ray} 
\vspace{-5cm}
%\address{W2B-096 Wellington Estate, DLF Phase 5, 122009 - Gurgaon - India}
\address{4/2 Carrer del Dr Fleming 4, Vilanova i la Geltrú 08800, Barcelona, Spain}
% optional, remove / comment the line if not wanted; the "postcode city" and and "country" arguments can be omitted or provided empty
%\phone[mobile]{+91-9971865758} 
\phone[mobile]{+34-696087836} 
% optional, remove / comment the line if not wanted
%\phone[fixed]{01234 123456}                    % optional, remove / comment the line if not wanted
%\phone[fax]{+3~(456)~789~012}                      % optional, remove / comment the line if not wanted
\email{siddhant.r98@gmail.com}                               % optional, remove / comment the line if not wanted
\homepage{www.linkedin.com/in/siddhant-ray (LinkedIn)}   
%\extrainfo{\href{https://github.com/Siddhant-Ray}{https://github.com/Siddhant-Ray (GitHub)}}                   
 % optional, remove / comment the line if not wanted
                % optional, remove / comment the line if not wanted
%\photo[64pt][0.4pt]{picture}                       % optional, remove / comment the line if not wanted; '64pt' is the height the picture must be resized to, 0.4pt is the thickness of the frame around it (put it to 0pt for no frame) and 'picture' is the name of the picture file
%\quote{Some quote}                                 % optional, remove / comment the line if not wanted

% to show numerical labels in the bibliography (default is to show no labels); only useful if you make citations in your resume
%\makeatletter
%\renewcommand*{\bibliographyitemlabel}{\@biblabel{\arabic{enumiv}}}
%\makeatother
\renewcommand*{\bibliographyitemlabel}{[\arabic{enumiv}]}% CONSIDER REPLACING THE ABOVE BY THIS

% bibliography with mutiple entries
%\usepackage{multibib}
%\newcites{book,misc}{{Books},{Others}}
%----------------------------------------------------------------------------------
%            content
%----------------------------------------------------------------------------------
\begin{document}
%\begin{CJK*}{UTF8}{gbsn}                          % to typeset your resume in Chinese using CJK
%-----       resume       ---------------------------------------------------------
\makecvtitle 
\vspace{-27pt}
%\begin{center}
    %I focus on learning, understanding and analyzing advanced engineering concepts in depth. My main areas of interest are machine learning, computer networks and distributed computing algorithms.  I am also enthusiastic about cloud computing and blockchain. I have a strong mathematical and programming background, with a keen interest to apply theoretical concepts learnt for solving practical problems.
%\end{center}

\vspace{-20pt}
% { \textbf{Expertise}: Marketing online et offline, Coordination des projets, Branding, Merchandising.\\
% \textbf{Projets réalisés}:http://gnt.globo.com/especiais/projetos-multitelas}

%\begin{multicols}{2}

\section{Education}

%\vspace{2pt}

%\begin{itemize}

{\cventry{2023 -- 2028}{The University of Chicago}{PhD in Computer Science}{}{}{ $\>$ $\>$ Advisor - Junchen Jiang and Nick Feamster}}
% - GPA -- /4
{\cventry{2020 -- 2022}{ETH Zürich}{MSc in Electrical Engineering and Information Technology}{}{}{ $\>$ $\>$ Advisor - Laurent Vanbever}}
% - GPA -- /6
{\cventry{2016 -- 2020}{VIT Vellore}{B.Tech in Electronics and Communication Engineering}{}{}{}}
%- CGPA -- /10
%\end{itemize}
\section{Research Projects}

% \vspace{2pt}

\begin{itemize}

\item{\cventry{Feb 2022 -- Aug 2022 }{\href{https://nsg.ee.ethz.ch/fileadmin/user_upload/thesis_proposal_packet_transformer.pdf}{Master Thesis at the Networked Systems Group }}{\href{https://github.com/Siddhant-Ray/Packet-Transformer}{Advancing Packet-Level Traffic Predictions with Transformers}}{ ETH Zurich}{}{\vspace{3pt} 
\begin{itemize}
\item Developed a Transformer based neural network which can efficiently learning network dynamics and make smarter packet-level forwarding decisions.
\item Using features from both the packet and network state, we capture the underlying patterns in the traffic in a task-agnostic pre-training phase followed by task-specific fine-tuning phases to leverage this learnt behaviour, make quicker updates and make better decisions.
\end{itemize}
}}

\item{\cventry{Mar 2021 -- Jun 2021}{\href{https://nsg.ee.ethz.ch/home/}{Semester Thesis at the Networked Systems Group }}{\href{https://github.com/Siddhant-Ray/FRR-P4-Super-Node-Prototype}{Towards a New Framework for Integration of Network Planes }}{ ETH Zurich}{}{\vspace{3pt} 
\begin{itemize}
\item Created a prototype for a new forwarding node in programmable networks i.e. a Super-Node to have better and more dynamic  control over forwarding decision in Layer-3 (L3) routers.
\item Our Super-Node consists of a traditional L3 router's control plane (CP) combined with a P4 programmable data plane (DP) for forwarding packets and we attempt accelerated forwarding and network convergence by leveraging the newly combined CP and DP.
\end{itemize}
}}

\item{\cventry{Feb 2021 -- May 2021}{\href{https://www.mins.ee.ethz.ch/index.html}{Semester Thesis at the Chair for Mathematical Information Science  }}{\href{https://github.com/Siddhant-Ray/Attentive-neural-networks-for-news-classification}{Attentive Neural Networks for News Classification}}{ ETH Zurich}{}{\vspace{3pt} 
\begin{itemize}
\item Developed a Transformer based neural network to classify a multi-class hierarchical, context-overlapping news dataset. 
\item Created a new statistical algorithm to detect context overlap in dataset classes, used it to reduce class label redundancy and demonstrated improvements of the classification model based on the algorithm's reduction
\end{itemize}
}}

\item{\cventry{Dec 2019 -- May 2020}{\href{https://cutt.ly/1nJxoKS}{International Journal of Mobile Network Design and Innovation}}{\href{https://github.com/Siddhant-Ray/Machine-Learning-Based-Cell-Association}{Machine Learning based Cell Association for 5G Communication Networks}}{ VIT Vellore}{}{\vspace{3pt} 
\begin{itemize}
\item Proposed a new cell association scheme to meet the ultra low latency, higher load and traffic needs of the 5G networks.
\item Proposed a Hidden Markov Model based learning algorithm followed by a Viterbi based decoding scheme, on the network’s telemetry data, to learn network parameters and select the optimal eNodeB for cell association.
 \end{itemize}
}}



\vspace{4pt}


% \item{\cventry{June 2018 -- Aug 2018}{\href{https://cutt.ly/MnJxdE3}{ResearchGate}}{A Comparative Analysis and Testing of Supervised ML Algorithms}{BlueStacks}{}{\vspace{3pt}
% \begin{itemize}
% \item  Studied and compared various supervised learning models and attempt to classify the best performing one based on varying comparison metrics
% \item This project was carried out on data captured from front-end customer preference analytics from the BlueStacks engine and use to develop a model to provide personalised customer experience.
% \end{itemize}
% }}

\end{itemize}
\section{Professional Experience}

% \vspace{2pt}

\begin{itemize}

    \item{\cventry{Sep 2022 -- Present}{Advanced Network Architecures Lab, UPC Barcelona}{{\href{https://www.craax.upc.edu/}{Researcher}}}{Barcelona, Spain}{}{\vspace{3pt} 
    \begin{itemize} 
    \item Working on data-driven and machine learning based resource management for cloud-edge computing systems.
    \end{itemize}
    }}

%\begin{itemize}

\item{\cventry{Oct 2021 -- Sep 2022}{\href{https://cutt.ly/fRzr0iJ}{Law, Economics, and Data Science Group, ETH Zurich}}{{\href{https://cutt.ly/CRlCu8t}{Graduate Research Assistant}}}{Zürich, Switzerland}{}{\vspace{3pt} 
\begin{itemize}
\item Research Assistant to Professor Dr. Elliott Ash and currently working on improving semantic labelling for text corpora using newer NLP models and sentence simplification and clustering for topic modelling.
\item Working on a paraphrase mining project to determine clusters of similar narratives in legal corpora and also using NLP models to capture underlying narratives in meat policy documents and analyse societal impacts and political discourse.
\end{itemize}
}}

\vspace{4pt}

\item{\cventry{May 2019 -- July 2019}{Capgemini Engineering}{Software Development Intern}{ Gurgaon, India}{}{\vspace{3pt} 
\begin{itemize}
\item Developed a K-Shortest Path Searching algorithm for the ONOS platform in a Java environment, automated and deployed using Maven.
\item The algorithm was subject to dynamic constraints in terms of network resources such as required edges and vertices etc. and was used for path calculation for deployed in in Software Defined Layer 2 VPNs.
\end{itemize}
}}

\vspace{4pt}

\item{\cventry{May 2018 -- July 2018}{BlueStacks}{Software Development Intern}{Gurgaon, India}{}{\vspace{3pt}
\begin{itemize}
\item  Worked on a machine learning algorithm to predict the App Engine's appropriate display screen based on the customer's past experiences.
\item  Developed an automation script for generating SVG cards for the App Engine's game front end and also worked on an address verification tool using the EasyPost API.
\end{itemize}
}}

\end{itemize}
% Publications from a BibTeX file without multibib
%  for numerical labels: \renewcommand{\bibliographyitemlabel}{\@biblabel{\arabic{enumiv}}}% CONSIDER MERGING WITH PREAMBLE PART
%  to redefine the heading string ("Publications"): \renewcommand{\refname}{Articles}

%\nocite{*} 
%\bibliographystyle{plain}
%\bibliography{cv}                        % 'cv' is the name of a BibTeX file

\section{Publications}

\bibliographystyle{plain}
\nobibliography{cv}
\renewcommand*\labelenumi{[\theenumi]}
\begin{enumerate}
    \item \bibentry{dietmuller2022new} [\href{https://arxiv.org/abs/2207.05843}{link}, \href{www.google.com}{code}]
    \item \bibentry{ray2020machine} [\href{https://www.inderscienceonline.com/doi/abs/10.1504/IJMNDI.2020.112622}{link}, \href{https://github.com/Siddhant-Ray/Machine-Learning-Based-Cell-Association}{code}]
\end{enumerate}

% Publications from a BibTeX file using the multibib package
%\section{Publications}
%\nocitebook{book1,book2}
%\bibliographystylebook{plain}
%\bibliographybook{publications}                   % 'publications' is the name of a BibTeX file
%\nocitemisc{misc1,misc2,misc3}
%\bibliographystylemisc{plain}
%\bibliographymisc{publications}                   % 'publications' is the name of a BibTeX file

\section{Skills}

\vspace{1pt}

\begin{itemize}

\item \textbf{Programming:} Python, C++, Java, Bash, SQL C, Rust

\vspace{1pt}

\item \textbf{Tools:} Linux,  P4 switches, ONOS controller, AWS, Google Cloud, Maven, MATLAB, NetSim, Cadence, Docker

\vspace{1pt}

\item \textbf{Frameworks:} Mininet, FRR suite, PyTorch, TensorFlow, Sklearn, NLTK, Flask, SciPy, Scapy, BeautifulSoup, ns3

\item \textbf{Natural Languages} -- English(C1) Hindi, Bengali, Deutsch(B1)

\end{itemize}
\section{Course Projects}

\begin{itemize}

\item \textbf{ {\href{https://github.com/Siddhant-Ray/Deep-Learning-Project-2021}{Investigating Possible Inductive Biases in Local Sparse Attention ViT Architectures Against Traditional CNNs}}} - ETH Zurich 2021 
\item \textbf{ {\href{https://github.com/Siddhant-Ray/NetworkSecurity-ACME-Project}{Automatic Certificate Management Environment framework}}} - ETH Zurich 2021  
\item  \textbf{ {\href{https://github.com/Siddhant-Ray/Advnet-Project-ETH}{Maximizing Cross Traffic Flows in a L2/L3 Network with Programmable Switches}}} - ETH Zurich 2020 
\end{itemize}
\section{Relevant Courses}
% \begin{itemize}
% \item \textbf{CS \& EE} - Advanced Computer Networks, System Security, Network Security, Distributed Computing, Discrete Event Systems, Computer Networks Seminar, Computer Networks, Operating Systems, Wireless Communication
% \item \textbf{Math \& ML} - Introductory Machine Learning, Deep Learning, Learning and Classification Theory, Mathematics of Data Science, Neural Network Theory, Linear Algebra
% \end{itemize}

\cventry{Graduate}{\mdseries Approximation Algorithms, Algorithms, Advanced Computer Networks, System Security, Network Security, Distributed Computing, Discrete Event 
    Systems, Networks Seminar, OS Seminar, Deep Learning, Learning Theory, Mathematics of
    Data Science, Neural Network Theory, Complexity Theory}{}{}{}{}
\cventry{Undergraduate}{\mdseries Computer Networks, Operating Systems, Wireless Communication, Linear Algebra}{}{}{}{}
\section{Honors and Awards}

\vspace{1pt}

\begin{itemize}

\item \textbf {{\href{https://github.com/FatjonZOGAJ/multilingual-text-analytics}{Winner at Datathon 2022, ETH Zurich (Microsoft Challenge)}}} -- (2022)
\item \textbf{Best Outgoing Student (SENSE department, VIT Vellore)} -- (2020)
\item \textbf {{\href{https://github.com/Siddhant-Ray/SlideEZ}{Runner-Up at VIT Hack 2019, VIT Vellore (Education Track)}}} -- (2019)
\item \textbf{Merit Scholarship for Academic Excellence, VIT Vellore } -- (2016-17, 2017-18, 2018-19)

\end{itemize}
\section{Leadership and Volunteering}

\vspace{1pt}

\begin{itemize}

\item \textbf{Technical Advisor at IETE VIT} -- (2019-20)
\item \textbf{President at Anokha NGO} -- (2018-19)
\item \textbf{Organizer at TEDx VIT Vellore} -- (2018-19)

\end{itemize}


%\end{multicols}

% Publications from a BibTeX file without multibib
%  for numerical labels: \renewcommand{\bibliographyitemlabel}{\@biblabel{\arabic{enumiv}}}% CONSIDER MERGING WITH PREAMBLE PART
%  to redefine the heading string ("Publications"): \renewcommand{\refname}{Articles}
%%\nocite{*}
%bibliographystyle{plain}
%\bibliography{cv}                        % 'publications' is the name of a BibTeX file

% Publications from a BibTeX file using the multibib package
%\section{Publications}
%\nocitebook{book1,book2}
%\bibliographystylebook{plain}
%\bibliographybook{publications}                   % 'publications' is the name of a BibTeX file
%\nocitemisc{misc1,misc2,misc3}
%\bibliographystylemisc{plain}
%\bibliographymisc{publications}                   % 'publications' is the name of a BibTeX file

%-----       letter       ---------------------------------------------------------

\end{document}


%% end of file `template.tex'.
